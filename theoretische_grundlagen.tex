\subsection{Standartmodell der Teilchenphysik} \label{standartmodell}
Das Standardmodell der Teilchenphysik ist eine theoretische Beschreibung der fundamentalen Bausteine der Materie und ihrer Wechselwirkungen.
Materie und Wechselwirkung kommen von drei Elementarteilchen, die Leptonen, Quarks und Bosonen. 
Leptonen und Quarks werden auch als Fermionen bezeichnet und haben einen Spin von $1/2$. 
Ebenfalls unterscheidet man zwischen drei Generationen von Leptonen und Quarks. 
Die Quarks unterscheiden sich in ihrem \textit{flavour}. 
Somit gibt es das up, down, charm, strange, top und bottom Quark. 
Desweiteren tragen Quarks die so genannte Farbladung.
Hier wird unterschieden zwischen rot, grün und blau, so wie den entsprechenden Antifarben. 
Die Leptonen treten, wie bereits erwähnt auch in drei Generationen auf. 
Jede Generation besteht aus einem Teilchen und dem dementsprechenden Neutrino. 
Die erste Generation umfasst das Elekktron $e^{-}$ und das Elektron-Neutrino $\nu_e$, die zweite Generation das Myon und das Myon-Neutrino $\nu_{\mu}$ und die dritte Generation das Tau $\tau$ so wie das Tau-Neutrino $\nu_{\tau}$.\\
\newline 
Die Bosonen sind die Vermittler der Wechselwirkungen. 
Bosonen haben einen ganzzahligen Spin und beseitzen, bis auf das W Boson $W\pm$, keine Ladung. 
So ist das Photon für die Übertragung der elektromagnetischen Wechselwirkungen,das W und Z Boson für die schwache Wechselwirkung und das Gluon für die starke Wechselwirkungen verantwortlich. 
Bei der starken Wechselwirkung kommen die Farbladungen wieder ins Spiel.


\subsection{Entstehung kosmischer Myonen}



\subsection{Dämpfung von Myonen Durch Blei}



\subsection{Myonen Silizium Interaktion}



\subsection{HV MAPS Sensor und Auslese}