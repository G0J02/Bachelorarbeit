\section{Motivation}
Der MuPix8 Sensor wurde ursprünglich für das Mu3e-Experiment entwickelt und wird in Zukunft auch Anwendung in dem $\overline{P}$ANDA-Experiment am Beschleunigunszentrum FAIR haben. 
Jedoch lässt sich der MuPix8 auch als Myonendetektor verwenden, was ihn somit interessant für die Methode der Myonentomografie macht. 
Diese bedient sich kosmischer Myonen welche in der Lage sind mehrere Kilometer an Fels zu durchdringen. 
Es kann sowohl die Absorption, als auch die Coulomb-Streuung der Myonen durch das Absorbermaterial beobachtet werden um selbst von großen Objekten oder Gesteinsschichten ein Bild zu erhalten. 
Diese Art um die Struktur und Zusammensetzung von Objekten zu studieren wird bereits in der Archäologie und der Untersuchung von Vulkanen verwendet und soll bald auch seinen Einsatzt in der Geothermie finden. \\
\newline
Diese Arbeit beschäftigt sich damit, in wie weit der MuPix8 Sensor für diese Aufgabe geeignet ist. 
Es wird die Absorption von Myonen, durch Blei untersucht um festzustellen ob bereits bekannte Phänomene in der  Detektion von Myonen Reproduziert werden können. 

